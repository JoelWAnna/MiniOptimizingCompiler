\documentclass[14pt]{article}
\author{Joel Anna<annajoel@pdx.edu>}
\usepackage{fancyhdr}
\usepackage{lastpage}
\pagestyle{fancy}
\lhead{\footnotesize \parbox{11cm}{Joel Anna} }
%\lfoot{\footnotesize \parbox{11cm}{\textit{2}}}
\cfoot{}
\rhead{\footnotesize Week 4 Report:  }
%\rfoot{\footnotesize Page \thepage\ of \pageref{LastPage}}
\renewcommand{\headheight}{24pt}
\begin{document}
\paragraph{CS401 Optimization Week 4 report} 
.\\
I spent this week preparing an attempt at a sweet source file for my global constant propagation algorithm, as well as documenting the steps taken in its creation. For the sweet source file, I believe that I was able to correctly format most of the new functions and classes, but am uncertain if my attempt to add new fields to an existing class is formatted correctly with sweet syntax. Additionally, I took steps to test and verify that the output from the algorithm works as intended and correctly optimized the program.
\paragraph{}
The remainder of the time spent this week was working through creating on paper IN and OUT sets for the basic blocks produced by the optimizer for each of the test programs. Inside a particular block, I understood from my reading that a 'Store' operation will kill any previous values generated by a 'Load' because of potential aliasing, a question that I had was for the case of multithreaded programs, does a Load operation have a non empty Gen set because another thread may write to that memory location.
After working through these problems by hand, for this week I would like to begin coding the generation and display of IN and OUT sets for use in future operations.
\paragraph{}
I have included a zip file of the MiniOptIm source files, a .sweet file both with and without the documentation that I wrote, and a zip file of Outputs from before and after the global optimization changes.
\end{document}